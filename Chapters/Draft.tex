
% \subsection{\ac{PON} Multi-Tenancy}

% In the rest of this section, we will look into the possible topologies and options for the sharable optical access networks, discussing their advantages and disadvantages. To further clarify, a shared \ac{PON} is meant to carry the traffic of multiple service providers, e.g., broadband, TV or mobile operators as its tenants. Each tenant is striving to meet the service specific requirements in different network layers, and they expect the \ac{PON} to also respect the demanded service differentiation.



\subsection{Blockchain and Smart Contracts}
Blockchain technology at its early days was introduced to address a particular problem regarding financial transactions where the robustness and operation of an entire ecosystem rely on the absolute trust placed upon a central authority (e.g., banking system). In other words, early use cases of blockchain technology were limited to utilizing the distributed ledger technology to replace central databases. However, after the introduction of smart contract technology, a much wider range of industries showed interest in using this novel technology to solve their problems. The smart contract technology makes use of the distributed network of the blockchain nodes and the consensus protocol to assure manipulation-proof execution of the application logic. The smart contract is an immutable piece of code that sits on top of the blockchain and invokes certain transactions that are pre-negotiated among the participants. In this paper, we use Hyperledger Fabric \cite{fabric}, a Linux Foundation Open Source permissionless blockchain framework, which is among the most adopted in both academic and industrial applications. 
The popularity of the Hyperledger Fabric is mainly due to its modular design, which allows the developers to customize each component of the blockchain network and also plug-in very critical parts of the platform such as the \textit{Consensus Mechanism}. 
An Hyperledger Fabric application consists of Organizations, Certificate Authorities  (CAs in charge of authorization and enrollment of new members), peer nodes (responsible for endorsing the transactions invoked by the Smart Contract), and orderers that assure correct ordering of the transaction records and packing them into blocks. The fault tolerance in the ordering process is guaranteed using the RAFT \cite{Ongaro:raft} consensus protocol. 
\subsubsection{Performance Evaluation and Metrics}
A Hyperledger Fabric blockchain application consists of various components (e.g., peers and orderers) that communicate with each other over a peer-to-peer network. These components are mainly implemented as micro-services hosted inside Docker containers. Therefore, the performance of a blockchain application relies on a number of various design parameters such as block size, block timeout, and state database \cite{8526892} that are outside the scope of this paper. However, in this paper we focus on the main performance determining parameter that is the transaction send rate (i.e., the rate in which the transaction proposals are sent to the blockchain application) and study how this will affect the two main following performance evaluation metrics:

\begin{enumerate}
    \item Transaction Throughput: The number of transactions that the blockchain application is able to process in a given time (a second).
    \item Transaction Latency: The amount of time it takes for a transaction to be processed by the blockchain from the time it was proposed/submitted by the application until its result/s are written on the ledger.
\end{enumerate}





% \subsection{\ac{PON} Virtualisation}
% In this section, We will introduce \ac{PON} virtualization as an enabler tool for multi-tenancy as it can provide the required flexibility for operator/service diversity in the network. As mentioned in the previous section, 

\subsubsection{OpenCord}

In addition to CORD, other researchers have also conducted some experimental work on the feasibility of network function virtualization in the context of converged wireless-optical networks.




Chengjun et al. \cite{7744444,7876161} have proposed slice scheduling scheme capable of assigning bandwidth slices to different tenants. Each slice of a \ac{PON} bandwidth resource is defined as an upstream XG-PON frame. The Slice Scheduler is an interface located between the network operators and the \ac{PON} transmission convergence layer to enable several operators to control their share of bandwidth resources in a time division multiplexing (TDM) manner. The second stage is the frame scheduling stage which lets each operator employ their customized \ac{DBA} to serve the \acp{ONU}. As mentioned by the authors there is a trade-off between the isolation and efficiency of bandwidth allocation, i.e. a completely isolated scheduling - referred to as static bandwidth resource sharing (static BRS) - would waste the excess bandwidth of the light loaded Operators. While another operator may need extra bandwidth and an efficient bandwidth allocation - referred to as Dynamic BRS - requires some interaction between two operators. The reported simulation results show that dynamic BRS, the sharing weights of operators will adapt to the real-time requirements of the operators and achieves high bandwidth efficiency.
However, The proposed frame-by-frame sharing of the \ac{PON} bandwidth will impose a minimum latency of a number of operators times frame duration for each operator's upstream transmission.

In \cite{6381701} three different unbundling strategies for realizing multi-operator \ac{GPON} were analyzed to choose the most suitable option for ease of market entrance for new network operators comparing their deployment costs. The first approach is moving back the \ac{PON} splitter to mimic an architecture similar to a P2P in which a single fiber is dedicated to each customer. The second strategy is replicating the access network, i.e. replicating the fiber deployment in the drop segment (the branches of the \ac{PON} tree) by dedicating a splitter to each operator. Moreover, the last option studied is upgrading the current TDM \ac{PON} to WDM PON. The cost analysis was done for scenarios with low to high density of customers in a square kilometer, and the conclusion is that upgrading to WDM is the most efficient strategy to perform LLU using \ac{GPON}.

