\chapter*{Methodology}
In this chapter we introduce and elaborate on the scientific methodology and experiment design used to evaluate and illustrate the contributions of this thesis.

To address the research questions raised in this research we have conducted a series of experimental studies. This includes the steps undertaken to evaluate and verify the claims made throughout this thesis. 

\newcommand{\RQa}{What are the implications and motivations for optical access network sharing?}
\newcommand{\RQb}{How to meet the technical requirements to enable fine-grained and dynamic optical access network sharing?}
\newcommand{\RQc}{Could a monetization market for excess resources incentivize the operators to share their excess resources with competitors?}
\newcommand{\RQd}{Could an economic robust double auction mechanism provide a market participation incentives for inter-operator network sharing?}
\newcommand{\RQe}{Could the blockchain technology be leveraged to address the lack of trust in centralized network sharing markets?}

\section{Research questions}
\begin{itemize}
\item \RQa
\item \RQb
\item \RQc
\item \RQd
\item \RQe


% \item What are the technical enablers for optical access sharing?
% \item What are the economic incentives for optical access sharing?

\end{itemize}

\section{Market Evaluation Methodology}
In this section we first define the criteria for economic robustness that is achieved by fulfilling a number of economic properties. Next we introduce the methods used to confirm the economic robustness of the proposed market mechanism including theoretical proof and game trees. Second using market simulations we illustrate the performance of the proposed model in terms of bandwidth utilization and market social welfare.


\subsection {Economic Properties (Economic-Robustness)}
\label{Method:subsec:Economic-Robustness}
The four essential principles of a desirable auction mechanism design includes optimal allocative efficiency (AE), incentive compatibility (IC), (ex-post) individual rationality (IR) and (ex-post weak) budget balance (BB) \cite{Krishna02}.

\begin{enumerate}
\item \textbf{Optimal Allocative Efficiency (AE):} The outcome allocation of the items maximizes the social welfare (i.e. the aggregate of all participants' utilities).
\item \textbf{Incentive Compatibility (IC):} An auction mechanism is IC when reporting the true valuation is a dominant strategy for all the traders, i.e. no trader can improve its utility gain from the market by reporting an untruthful value. This is also referred to as by "Truthfulness" and "Strategyproofness" in the literature.
\begin{Definition}
Dominant strategy: In game theory, a strategy is dominant if regardless of what other players do there is no alternative strategy to be played that will bring more utility to the player.
\end{Definition}
IC provides strong participation incentives for the traders by reducing the participation cost. The reasons to eliminate strategic behavior from the market are as follows:
\begin{enumerate}
\item Strategic behavior of the traders makes the market very complicated to analyze. Especially for a market such as double-auction multi-item market in which there is competition both between the same type of the traders (i.e. seller/seller or buyer/buyer) and opposing type of traders (i.e. seller/buyer) and there is an incentive for them to strategize through untruthful value/quantity reporting to achieve a higher utility.

\item Strategic behavior can impose a substantial social cost on the market as it promotes competitive strategizing. The traders would spend resources to acquire more information about the market and their competitors' preferences, and this consequently will negatively affect their market power, i.e. asks/bids.
\end{enumerate}
\item \textbf{Ex-post Individual Rationality (IR):} All traders have non-negative utility if they participate in the market.
\item \textbf{Ex-post weak Budget Balance (BB):} The auctioneer does not run at a negative utility. The mechanism is referred to as weakly budget-balanced if the auctioneer does not get a negative utility but it may have a positive utility, and strongly budget-balanced, i.e. the auctioneer's is exactly zero. Our desired mechanism is weakly budget-balanced as the auctioneer will get the market surplus as its operation fee.
\end{enumerate}
\subsection{Theoretical Proofs}

Considering that the number of cases are finite and limited we have used \textit{Proof by Exhaustion} method to illustrate the satisfaction of the desired economic properties. We conduct a case analysis to study every possible outcome of the market players' strategic behavior. This is to assure that none of the market players could use manipulative techniques to undermine the market and achieve higher profits from it. Such manipulative techniques might include artificial intelligence assisted bidding strategies that could lead to further challenges. 

\subsection{Game Trees}
For further clarification we use game trees to visually represent the theoretical proofs. Game trees offer a visual alternative to illustrate the case by case outcome of the players' strategy in the game. 
\subsection{Market Simulation}
\label{Method:subsec:simul}
While theoretical proofs can be used to assure satisfying \ac{IC}, \ac{BB}, and \ac{IR}. However, to test and evaluate \acl{AE} it is necessary to observe the market and gather information on certain parameters. This information is then used to compare the efficiency of the market mechanism proposed in this thesis with the state-of-the-art.
Therefore we have developed a market simulation tool that gathers relevant parameters that are necessary in investigating the market performance. The market simulator can support various auction mechanisms and generate reproducible results on the following factors:
\begin{itemize}
    \item Market Utility Distribution: The share of sellers/buyers from the market.
    \item Social welfare: The aggregate utility of all the market participants.
    \item Network utilization: Bandwidth utilization of the \ac{PON}.
\end{itemize}

All of the above 
allow comparitive analysis

is capable of conducting various 


limitations

\textbf{
}


Name
connect01
Login
connect01.tchpc.tcd.ie
Nodes
1
Processors
44
Processors per node
44
Architecture
64-bit
OS
Scientific Linux 6x

% \section{Blockchain Benchmark}
% Blockchain technology offers novel solutions addressing a range of problems such as centralized trust and single points of failure.

% \torephrase{As enterprises begin to adopt blockchain technology to solve business problems, there are valid concerns if blockchain applications can support
% the transactional demands of production systems.}

% It is important to note that blockchain on redundant procecesses to provide these soloutions. For example, the a simele blockchain-based record-keeping application requires $n \times$

\section{Blockchain Performance Evaluation}
A blockchain application operates over an underlying network of different components. In \cite{10.1007/978-3-030-16946-6_5,10.1007/978-3-662-58820-8_18} the authors have proposed different distributed auction mechanisms over an Ethereum network and analyzed the performance/cost of their implementation in terms of estimated gas costs and time efficiency. 
The main objective of a blockchain application is to handle a number of transactions that are submitted by the participants and proceed to the verification and ordering process, leading to the generation of a block and the transaction outcome being written on the ledger.
In the context of Hyperledger blockchain frameworks the performance of the application is closely tied to the performance of each component (e.g., peers, orderers, containers, etc.) and the network that interconnects them. The performance of a blockchain application/network can be measured using the following metrics:


\begin{itemize}
    \item Transaction Throughput, measured in transactions per second (TPS):
    The number of transactions that are processed by the blockchain and written on the ledger in a given second.
    
    \begin{equation} \label{eq1}
    \begin{split}
    Transaction\,Throughput = \frac{Total\,Transactions}{Total\,time\,in\,seconds}
    \end{split}
    \end{equation}

    \item Transaction Latency:
    The amount of time taken from the moment when a transaction is submitted until the moment when it is confirmed and available on the blockchain. This includes the propagation time and the processing time due to the consensus/ordering mechanism.
    
    \begin{equation} \label{eq2}
    \begin{split}
    Transaction\,Latency = t_{Confirmation} - t_{Submission}
    \end{split}
    \end{equation}


    \item Computing Intensity:
    The amount of computing resources consumed by the blockchain throughout the operating time, including the processing power, memory, storage, I/O and network. This metric is of great importance as it could determine the cost efficiency of a blockchain application. Furthermore, besides the capital expenditure for providing the computing capacity, blockchain networks could require huge amounts of energy to operate. Therefore, the computing intensity would also affect the operation costs of the blockchain.
    

\end{itemize}


The performance of three major blockchain frameworks is compared in Table. \ref{tab:bc-perform}.
A more in-depth study of the performance metrics and evaluation methods is presented by the Hyperledger Performance and Scale Working Group  \cite{hgperf,pswg}.

\begin{table}[htbp]
  \caption{Performance of Blockchain Frameworks} 
  \label{tab:bc-perform}
  \small % text size of table content
  \centering % center the table
  \begin{tabular}{lcccccccr} % alignment of each column data
  \toprule[\heavyrulewidth]\toprule[\heavyrulewidth]
  Platform/Metric & Bitcoin    & Ethereum     &  Fabric \\ \hline
  \midrule
    Average Latency  & $\approx 10\, Min$ & $\approx 12.5\, Sec$ & $\approx MilliSec$      \\
    Throughput (TPS) & 7   & 10 - 30    & 20,000 \cite{Gorenflo_2019}          \\
  \bottomrule[\heavyrulewidth] 
  \end{tabular}
\end{table}



\subsection{Deployment}
Container technology
Docker Swarm Container orchestration
Google Cloud
Maybe snapshots of the code


\subsection{Caliper}

Caliper is a blockchain benchmark framework which allows users to measure the performance of a specific blockchain implementation with a set of predefined use cases. Caliper will produce reports containing a number of performance indicators, such as TPS (Transactions Per Second), transaction latency, resource utilisation etc. The intent is for Caliper results to be used as a reference in supporting the choice of a blockchain implementation suitable for the user-specific use-cases. Given the variety of blockchain configurations, network setup, as well as the specific use-cases in mind, it is not intended to be an authoritative performance assessment, nor to be used for simple comparative purposes (e.g. blockchain A does 5 TPS and blockchain B does 10 TPS, therefore B is better). The Caliper project references the definitions, metrics, and terminology as defined by the Performance and Scalability Working Group (PSWG).

Key Characteristics
A unified blockchain benchmark framework. We provide a common layer to integrate with major existing blockchain framework/platforms, so that the same benchmarks can be run for different blockchain systems Some benchmark test environment will be provided to help different people run tests under the same environment, blockchain management tools like Hyperledger Cello could be integrated later to deploy and operate the environment. Also, users can use their existing environment and configure Caliper to run the test under the environment.

A commonly accepted definition of performance indicators. You cannot compare an apple and a pear directly unless some common criteria are set. We will work closely with PSWG to provide a common definition of performance indicators that users care about, such as TPS, latency, resource utilization, etc.

A set of commonly accepted benchmark cases. The goal of Caliper includes providing a set of easy-understandable benchmark cases so that each blockchain solution can be compared in various scenarios. This calls for much collaboration from PSWG, Requirement WG and other WG in Hyperledger community as well as blockchain practitioners to cover as many use cases that are of user’s interest as possible.


Prometheus Monitor
Prometheus is an open-source systems monitoring and alerting toolkit that scrapes metrics from instrumented jobs, either directly or via an intermediary push gateway for short-lived jobs. It stores all scraped samples locally and runs rules over this data to either aggregate and record new time series from existing data or generate alerts. Grafana or other API consumers can be used to visualize the collected data.





\section{PON STUFF}

\subsection{Blockchain and Smart Contracts}
Blockchain technology at its early days was introduced to address a particular problem regarding financial transactions where the robustness and operation of an entire ecosystem rely on the absolute trust placed upon a central authority (e.g., banking system). In other words, early use cases of blockchain technology were limited to utilizing the distributed ledger technology to replace central databases. However, after the introduction of smart contract technology, a much wider range of industries showed interest in using this novel technology to solve their problems. The smart contract technology makes use of the distributed network of the blockchain nodes and the consensus protocol to assure manipulation-proof execution of the application logic. The smart contract is an immutable piece of code that sits on top of the blockchain and invokes certain transactions that are pre-negotiated among the participants. In this paper, we use Hyperledger Fabric \cite{fabric}, a Linux Foundation Open Source permissionless blockchain framework, which is among the most adopted in both academic and industrial applications. 
The popularity of the Hyperledger Fabric is mainly due to its modular design, which allows the developers to customize each component of the blockchain network and also plug-in very critical parts of the platform such as the \textit{Consensus Mechanism}. 
An Hyperledger Fabric application consists of Organizations, Certificate Authorities  (CAs in charge of authorization and enrollment of new members), peer nodes (responsible for endorsing the transactions invoked by the Smart Contract), and orderers that assure correct ordering of the transaction records and packing them into blocks. The fault tolerance in the ordering process is guaranteed using the RAFT \cite{Ongaro:raft} consensus protocol. 
\subsubsection{Performance Evaluation and Metrics}
A Hyperledger Fabric blockchain application consists of various components (e.g., peers and orderers) that communicate with each other over a peer-to-peer network. These components are mainly implemented as micro-services hosted inside Docker containers. Therefore, the performance of a blockchain application relies on a number of various design parameters such as block size, block timeout, and state database \cite{8526892} that are outside the scope of this paper. However, in this paper we focus on the main performance determining parameter that is the transaction send rate (i.e., the rate in which the transaction proposals are sent to the blockchain application) and study how this will affect the two main following performance evaluation metrics:

\begin{enumerate}
    \item Transaction Throughput: The number of transactions that the blockchain application is able to process in a given time (a second).
    \item Transaction Latency: The amount of time it takes for a transaction to be processed by the blockchain from the time it was proposed/submitted by the application until its result/s are written on the ledger.
\end{enumerate}

Hyperledger Caliper \cite{caliper} is one of the sub-projects under the Linux Foundation's blockchain initiative, which is designed to provide a benchmarking and performance evaluation tool for various blockchain frameworks (e.g., Sawtooth, Fabric, Etherium and more.). In this paper, we use Hyperledger Caliper to evaluate the performance of our proposed distributed \ac{PON} sharing market, and the results of these experiments are reported in section \ref{sec:results}.
