\chapter*{Summary}



Multiple parallel trends including the growing number of internet reliant devices/services, increasing internet penetration rates, and the continuing popularity of bandwidth-hungry multimedia content contribute to the exponential surge of internet traffic. The combination of these trends could imply a considerable increase in network infrastructure investment for the telecom and broadband operators, often leading to a rise in end-user fees. In addition, the high cost of initial investment could escalate the market barriers to entry for the innovative service providers incapable of deploying their own network infrastructure. In this thesis, we explore if and how enabling optical access network sharing could cultivate new network ownership and business models that simultaneously keep the end-user subscription fees low and facilitate the market entry for the smaller service providers. We aim to identify and address the technological and economic barriers of optical access network sharing.
The broad scope of this thesis concerns the inter-operator sharing of optical access networks which connect the end-users to the operators' network in the last-mile. The access segment of the communications network is recognized to be the most costly due to its deployment scale. Therefore, a reduction in cost in the access will have a multi-fold impact on the overall capital expenditure for network deployments. The thesis focuses in particular on \acp{PON} as the most widespread type of optical access networks. 

The central argument of the present research is that network infrastructure/resource sharing has the potential to reduce the capital and operational expenditure of the network operators. This will allow for more competition as the market entrance cost decreases.

We first address the lack of tenant operators' adequate control over the shared resources in a multi-tenant \ac{PON} as a technological barrier. We provide a solution to strengthen the network operators' control over their share of the network in a multi-tenant \ac{PON}. This is made possible by allowing the operators to schedule the transmission over the network using tailored algorithms to meet their requirements (e.g., latency and throughput). The thesis argues that providing a virtual (software) instance of the \ac{DBA} algorithm as opposed to the inflexible hardware implementation first, enables the coexistence of various services on the \ac{PON} and second, improves the overall utilization of the network capacity. 


While the virtualization of the \ac{DBA} removes the technical barrier for the inter-operator resource sharing, it does not come with a natural incentive for the operators to share their resources with competitors. Therefore we tackle the lack of incentive for sharing excess network capacity in \acp{PON} by providing monetary compensation in return for sharing. We model the multi-tenant optical access network with multiple coexisting operators as a market where they can exchange their excess capacity. We propose a sealed-bid multi-item double auction to enable capacity trading between the network operators. Through mathematical proof and market simulation/visualization, we prove that the proposed auction mechanism meets the essential requirements for an economic robust market mechanism (e.g., incentive compatibility, individual rationality, and budget balance). This provides trusted market conduct in the presence of a central authority (e.g., the public infrastructure provider) that all the operators trust. 

The shift in the market ownership models motivated us to explore an alternative scenario where the public sector is not involved in infrastructure management; hence no central entity is to be trusted to conduct the market mechanism. Therefore, we argue that the blockchain technology can be exploited to hold the market in a distributed fashion as an alternative to centralized control. To analyze the feasibility of such a distributed market, we developed smart contracts that implement an auction algorithm capable of allocating the resources to the participants without a central trusted market mediator. We use the open-source framework Hyperledger Fabric to develop the blockchain application. The nodes of the blockchain network are then distributed across multiple cloud-hosted virtual machine instances to allow more realistic and precise experimentation. We use common metrics such a transaction latency and throughput to evaluate the performance of the designed marketplace application. Furthermore, we study the computing resources required to run the blockchain application.

% Throughout this thesis we will address the following questions:
% \begin{itemize}
%     \item What are the implication and motivations for optical access sharing?
%     \item What are the technical enablers for optical access sharing?
%     \item What are the economic incentives for optical access sharing?
% \end{itemize}


% The core themes of my thesis are as follows:
% \begin{enumerate}

% \item Strengthening the virtual network operators' control over their share of the network in a multi-tenant PON. This is possible by allowing the operators to schedule the transmission over the network using their tailored algorithms to meet their requirements (latency and throughput). The thesis argues that providing a virtual (software) instance of the Dynamic Bandwidth Allocation (DBA) algorithm as opposed to the inflexible hardware implementation first enables coexistence of various services on the PON and second, improves the overall utilisation of the network capacity. I have used network simulation as the methodology to prove our claims. The comparative analysis showed that the above claims are valid.

% \item The second argument is that the lack of incentive for sharing secondary network capacity in PONs can be addressed by providing monetary compensation in return for sharing. I claimed that a sealed-bid multi-item double auction could be designed and used to trade capacity between network operators. Through mathematical proof and market simulation/visualization, I proved that my proposed auction algorithm meets the essential requirements for an economic robust market mechanism ( incentive compatibility, individual rationality, and budget balance ). This provides trusted market conduct in the presence of a central authority (e.g., the public infrastructure provider) that all the operators trust.


% \item Considering the shift in the market ownership models where the public sector is not involved in infrastructure management, no central entity is to be trusted to conduct the market mechanisms. Therefore, I argue that the blockchain technology can be exploited to hold the market in a distributed fashion as an alternative to the centralised control. To analyse the feasibility of such a distributed market, I developed smart contracts that implement an auction
% algorithm which will be capable of allocating the resources to the participants without a central trusted market broker. I use an open-source framework called Hyperledger Fabric to design the blockchain application. The nodes of the blockchain network are then distributed across multiple cloud-hosted virtual machine instances to allow more realistic and precise experimentation. I used common metrics such a transaction latency and throughput to evaluate the performance of the designed marketplace application. Furthermore, I studied the computing resources required to run the blockchain application.
% \end{enumerate}

% We study the technical requirements for sharing the resources and infrastructure of optical access networks among multiple network operators. These operators provide various  services on top of the same network infrastructure. Each of these services has various requirements that ought to be met by supplying them with sufficient control over their portion of the network. One of the main control functions in a Passive Optical Network (PON) is the Dynamic Bandwidth Allocation (DBA) mechanism that influences vital performance factors such as latency and Jitter. PON networks are considered as one of the primary candidates for deploying fronthaul section of the Centralized Radio Access
% Networks (C-RAN). However, the current hardware implementation of DBAs fails to accommodate the stringent latency requirements of the C-RAN fronthaul. To address this, as the first contribution of this thesis, we propose a virtual DBA mechanism which would enable the co-existing operators to implement their flavour of the DBA without interfering with others'. In the second contribution of this thesis, we investigate the economic incentives of different market players in an optical access network sharing ecosystem. We reason that since the cohabiting network operators are business competitors, they lack any natural incentive to share excess resources. Therefore we propose to create the inter-operator sharing incentive through monetary compensation.
% We developed an economic-robust and efficient sharing platform to enhance network utilization. We have designed a centralized double auction mechanism to facilitate the trading of excess resources. We prove that the proposed mechanism satisfies the fundamental economic properties (e.g., incentive compatibility, individual rationality and weak budget balance).
% As the third contribution, we have tackled the lack of trust in a central authority to conduct the auction and manage the market. We designed a distributed market mechanism using the blockchain and smart contract technology to address this problem. New network ownership models challenge the existence of a central authority trusted by all of the network beneficiaries to manage the market. The proposed distributed market aims to address this lack of trust by eliminating the centralised, monopolistic control of the resource sharing market. Methodology: Real-world solution development using open-source Hyperledger Fabric platform and practical deployment on cloud-hosted servers.


