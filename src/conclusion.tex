\section{Conclusions}
\subsection{Research Question 1:}
\subsection{Research Question 2:}
\subsection{Research Question 3:}
\section{Future Work}



\subsection{Quality Assurance and Enforcement}
A slice of the network comes with certain qualitative guarantees, often in the form of a pre-negotiated \ac{SLA}. The \ac{SLA} has to provide a detailed description of the expected reliability, availability and other performance metrics of the service and the actions to be taken in case of one of the parties breaching the agreement (i.e., the supplier not meeting the terms of SLA). The authors in \cite{SLA-Slicing} have introduced a number of metrics regarding the slice-based network \acp{SLA} including throughput, penalty, cost, revenue, profit, and QoS related metrics. However, the enforcement of these \acp{SLA} remains an open issue as no automated process has been proposed to address the problem. In the next section, we will discuss how smart contract technology can provide an automatic approach towards enforcement of \acp{SLA}.