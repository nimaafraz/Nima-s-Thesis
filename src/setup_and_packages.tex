%%%%%%%%%%%%%%%%%%%%%%%%%%%%%%%%%%%%%%%%%%%%%%%%%%%%%%%
%%%%%% PACKAGES %%%%%%%%%%%%%%%%%%%%%%%%%%%%%%%%%%%%%%%
%%%%%%%%%%%%%%%%%%%%%%%%%%%%%%%%%%%%%%%%%%%%%%%%%%%%%%%

%%% REQUIRED PACKAGES
\usepackage[british]{babel}  % for American English
\usepackage{setspace}         % for double and onehalf spacing

\usepackage{soul,color} %for using hl to highlight thing in the thesis
% \usepackage{subfig}
\usepackage{booktabs}
\usepackage{ragged2e}
%%% LISTS AND TABLES
\usepackage{paralist}         % to have more compact lists
%\usepackage{expdlist}         % to adjust layout of description environment
%\usepackage{array}           % for enhancing tables
%\usepackage{multirow}        % for centering over rows in tables
\usepackage{colortbl}         % for colored rows or columns in tables
\usepackage[table]{xcolor}

\definecolor{LightCyan}{rgb}{0.88,1,1}
\definecolor{Red}{rgb}{1,0,0}
\definecolor{Green}{rgb}{0,1,0}

\definecolor{DarkGreen}{rgb}{0,0.6,0}

\definecolor{Blue}{rgb}{0,0,1}
\definecolor{Orange}{rgb}{1,0.45,0}


\definecolor{White}{rgb}{1,1,1}

\newcolumntype{w}{>{\columncolor{White}}c}

%%% MATH
\usepackage{amsmath}          % enhance mathematics
\usepackage{amsfonts}         % some further mathematic symbols
\usepackage{amsthm}           % for designing theorems
\usepackage{bm}               % provides bold math \bm{} command (vectors)
%\usepackage{mathrsfs}          % provides math script font \mathscr{}


%%% GRAPHICS, FIGURES, FLOATS
\usepackage{graphicx}
\usepackage{float}
\graphicspath{{../figures/}{./figures/}{./figures/stochastic/}{../figures/stochastic/}}
\DeclareGraphicsExtensions{.pdf,.jpeg,.png,.eps,.jpg}
        % for graphics
%\usepackage{psfrag}           % replace text in eps graphics
\usepackage{placeins}         % to place all pending floats before a new section begins
% \usepackage{subfigure}        % multiple figures in one figure

% \usepackage[font=small]{caption}

%%% FONTS, FONTENCODING, COLOUR
%\usepackage{textcomp}         % many useful symbols, including copyright and trademark
%\usepackage{times}           % Times Font
%\usepackage{palatino}        % Palatino Font
\usepackage[T1]{fontenc}      % two use only nice Type 1 fonts (vector fonts)
% \usepackage[latin1]{inputenc} % use non-ASCII characters in text directly
\usepackage[utf8]{inputenc}
\usepackage{color}			  % for coloured text
\usepackage{eurosym}					% For euro symbols

%%% PDF STUFF
\usepackage[
pdfpagelabels,                % uncomment this line for real page numbers in PDF
breaklinks,
plainpages=false,
pdfstartview=FitH,
hyperfootnotes=false,
hidelinks]{hyperref}         % best with pdf output (links, bookmarks)
% \usepackage{breakurl}         % for nice URLs with hyperref
\usepackage{url}             % for nice URLs without hyperref

\usepackage{pdfpages}
%%% SECTIONING, BIBLIOGRAPHY, INDEX, LAYOUT
\usepackage[printonlyused,nohyperlinks]{acronym}  					% nice list of acronyms
%There is a problem with the implemenatation of the acronym package,
%whenever a parameter is given the font is changed.

\usepackage[acronym,nomain]{glossaries}
\usepackage[numbers,sort&compress]{natbib}    % for natural Bibliographies
% \usepackage[resetlabels,labeled]{multibib}
% \usepackage{biblatex}

%% The new list's label is "New" and will be titled "The other list".
%% To put cites into this list, use \citeNew.
% \newcites{New}{My Publications}
% \usepackage{usebib}




\usepackage[notbib]{tocbibind}        				% lists, indexes, bib in contents
\usepackage[titletoc,page,header]{appendix}   % for a nicer appendix
\usepackage{makeidx}         									% for creating indexes
\usepackage{fancyhdr}            							% for nice headers and footers
\usepackage[sf,bf]{titlesec}         % formating headings
%\usepackage{titletoc}             % formating table of contents
\usepackage{tocloft}      %% Control the formatting for TOC

\usepackage{cleveref} %provides command cref for clever referencing

\usepackage{mathtools}
\usepackage{multirow}
\usepackage{threeparttable}
\usepackage{pifont}
\usepackage{dsfont}
\usepackage{amsfonts,amsmath,amssymb}
\usepackage{stfloats,psfrag}
% \usepackage[algoruled,medskip,dontprintsemicolon,linesnumbered,Algorithm]{algorithm}
% \usepackage[noend]{algpseudocode}
\usepackage[percent]{overpic}
\usepackage[export]{adjustbox}

\usepackage{enumitem}

\DeclarePairedDelimiter\ceil{\lceil}{\rceil}

\usepackage{comment}
% page layout
\usepackage[a4paper,
%hdivide={ ,16cm,1.8cm},
bindingoffset=1.8cm,
%scale=0.8,
%hmarginratio=1:1,
%right=2cm,
left=2cm,
top=1.8cm,
headheight=15pt,
%headsep=16pt,
%footskip=28pt,
%textheight=22.6cm,
hcentering,
vcentering,
includeheadfoot]{geometry}

% for better lists of citations (dont use cite package if the citations
% numbers in the
% text should be links to the bibliography - with hyperref)
% \usepackage[noadjust]{cite}       % sorted and compressed list of num cites [3-6]
%\usepackage[export]{splitbib}      % for sections in bibliography (AFTER HYPERREF!)
% \usepackage[numbers,sort&compress]{natbib} % Natbib bibliography
% \usepackage{hypernat}
\usepackage{bibentry}
% \nobibliography*


%\usepackage{IEEEtrantools}   % for IEEE bib control entry
\usepackage[redeflists]{IEEEtrantools}


%%%%%%%%%%%%%%%%%%%%%%%%%%%%%%%%%%%%%%%%%%%%%%%%%%%%%%%
%%%%%% PERSONAL SETTINGS %%%%%%%%%%%%%%%%%%%%%%%%%%%%%%
%%%%%%%%%%%%%%%%%%%%%%%%%%%%%%%%%%%%%%%%%%%%%%%%%%%%%%%

\newcommand{\authorname}{Nima~Afraz}   % Author

% adjust colors for links to your liking (dark colors are better for S/W print)
\definecolor{darkblue}{rgb}{0,0,0.5}
\definecolor{darkgreen}{rgb}{0,0.5,0}
\definecolor{darkred}{rgb}{0.5,0,0}
\hypersetup{
   pdftitle={Nima-Afraz-PhD-Thesis},
   pdfauthor={Nima~Afraz, Supervisor: Prof. Marco Ruffini},
   pdfsubject={PhD Thesis},
   pdfkeywords={},
   pdfview=FitH,
   pdfstartview=FitH,
   bookmarksnumbered  
   % uncomment this line if you want to print in copyshop!!!
   % the links (biblinks, acronyms, ..) will not be coloured then
   %,colorlinks,linkcolor=darkred,citecolor=darkgreen,urlcolor=darkblue
}

\usepackage{etex} %we need this package to resolve problems if too many packages are used
\usepackage{color}
\usepackage{tikz}
\usepackage{chronology} %we have to use our customized package due to problems with the used version of tikz, and conflict with units package
\usepackage{units}            % for nicely typset units \unit[10]{kHz}
\usepackage{footnote}
\usepackage{hhline}
%\usepackage{subcaption}
%\usepackage[caption=false,font=footnotesize]{subfig}

\newcommand{\Mc}{\mathcal{M}}
\newcommand{\Lc}{\mathcal{L}}
\newcommand{\Fc}{\mathcal{F}}
\newcommand{\Tc}{\mathcal{T}}
\newcommand{\Pc}{\mathcal{P}}
\newcommand{\Sc}{\mathcal{S}}
\newcommand{\Cc}{\mathcal{C}}
\newcommand{\Nc}{\mathcal{N}}
\newcommand{\Dc}{\mathcal{D}}
\newcommand{\Wc}{\mathcal{W}}
\newcommand{\Bc}{\mathcal{B}}
\newcommand{\Qc}{\mathcal{Q}}
\newcommand{\Ac}{\mathcal{A}}
\newcommand{\Pb}{\mathbb{P}}
\newcommand{\Eb}{\mathbb{E}}
\newcommand{\Rb}{\mathbb{R}}
\newcommand{\sinr}{\mathrm{SINR}}
\newcommand{\sir}{\mathrm{SIR}}
\newcommand{\Rs}{\mathbb{R}^2}
\newcommand{\Rp}{\mathbb{R}_+}
\newcommand{\Nb}{\mathbb{N}}
\DeclareMathOperator*{\argmax}{\arg\!\max}
\DeclareMathOperator*{\argmin}{\arg\!\min}
\newcommand{\mcmint}{\int^{r_{max}}_{r_{min}}}
\newcommand{\Phio}{\Phi\setminus \{o\}}
\newcommand{\Phiib}{\Phi_i^\prime\setminus b(o,r)}
\newcommand{\hPhi}{\hat{\Phi}}
\newcommand{\Gxo}{G^{[x]}_0[v]}
\newcommand{\tl}{\tilde{l}}
\newcommand{\Bf}{\mathfrak{B}}
\newcommand{\Ic}{\mathcal{I}}
\newcommand{\Xc}{\mathcal{X}}
\newcommand{\Pd}{\mathbb{P}}
\newcommand{\hSc}{\hat{\Sc}}
\newcommand{\hScr}{\hat{\Sc}_{\texttt{res}}}
\newcommand{\dt}{R}
\newcommand{\rev}{Q}

\newcommand*{\myprime}{^{\prime}\mkern-1.2mu}
\newcommand*{\mydprime}{^{\prime\prime}\mkern-1.2mu}
\newcommand*{\mytrprime}{^{\prime\prime\prime}\mkern-1.2mu}

\newcommand{\tp}{\theta\myprime}
\newcommand{\bp}{\beta\myprime}

\newcommand{\nash}{\text{NE}}
\newcommand{\nbs}{\text{NBS}}
\newcommand{\kss}{\text{KSS}}
\newcommand{\ubs}{\text{UBS}}
\newcommand{\tne}{\theta_\nash}
\newcommand{\bne}{\beta_\nash}
\newcommand{\tnbs}{\theta_\nbs}
\newcommand{\bnbs}{\beta_\nbs}
\newcommand{\hbne}{\hb_\nash}
\newcommand{\hb}{\hat{\beta}}
%\newcommand{\bp}{\beta^\prime}
%\newcommand{\hpm}{\hat{p}_\mno}
\newcommand{\hpm}{c_m}

\newcommand{\yale}[1]{``#1''}

\newcommand{\Phin}{\Phi_n}
\newcommand{\hPhin}{\hat{\Phi}_n}
\newcommand{\Phim}{\Phi_m}
\newcommand{\Rss}{R_{n}}
\newcommand{\rs}{d^\star}
\newcommand{\xs}{x^\star}
\newcommand{\lambdap}{\lambda^\prime}

%\newcommand{\Cc}{\mathcal{C}}
\newcommand{\Zp}{\mathbb{Z}_+}
\newcommand{\Ed}{\mathds{E}}
\newcommand{\mno}{\text{MNO}}
\newcommand{\ott}{\text{OTT}}

\newcommand{\Vbf}{\mathbf{V}}
\newcommand{\Wbf}{\mathbf{W}}
\newcommand{\Abf}{\mathbf{A}}

\newcommand{\Zf}{\mathfrak{Z}}
\newcommand{\Af}{\mathfrak{A}}
\newcommand{\Cf}{\mathfrak{C}}

%% This is to make a diagonal line in a cell using multirow and multicolumn
\usepackage{diagbox}
\usepackage{rotating}

%Checkmark and Xmark
\newcommand{\cmark}{\ding{51}}%
\newcommand{\xmark}{\ding{55}}%

% footnote asterisk
\newcommand{\footstar}[1]{$^*$ \footnotetext{$^*$#1}}
% footnote double asterisk
\newcommand{\footdoublestar}[1]{$^{**}$ \footnotetext{$^{**}$#1}}
% footnote triple asterisk
\newcommand{\foottriplestar}[1]{$^{***}$ \footnotetext{$^{***}$#1}}

\newcommand{\indicator}[1]{\mathds{1}_{\left[ {#1} \right] }}
% \newcommand{\Fig}[1]{Fig.~\ref{fig:#1}}
% \newcommand{\cFig}[1]{\Cref{fig:#1}}
%\newcommand{\cFigrange}[1]{\Crefrange{fig:#1}}
\newcommand{\Sec}[1]{Sec.~\ref{sec:#1}}
% \newcommand{\Eq}[1]{Eq.~(\ref{eq:#1})}
\newcommand{\Alg}[1]{Alg.~\ref{alg:#1}}
% \newcommand{\Lemma}[1]{Lemma~\ref{lem:#1}}
% \newcommand{\Theorem}[1]{Theorem~\ref{thm:#1}}
% \newcommand{\Corollary}[1]{Corollary~\ref{cor:#1}}
\newcommand{\App}[1]{Appendix~\ref{app:#1}}
\newcommand{\LineAlg}[1]{Line~\ref{line:#1}}
\newcommand{\Cpt}[1]{Chapter~\ref{cpt:#1}}
\newcommand{\Prop}[1]{Property~\ref{prop:#1}}
\newcommand{\Tab}[1]{Tab.~\ref{tab:#1}}

\newtheorem{property}{Property}
% \newtheorem{lemma}{Lemma}
\newtheorem{proposition}{Proposition}
% \newtheorem{theorem}{Theorem}
% \newtheorem{corollary}{Corollary}

\newcommand{\chapquote}[3]{\begin{quotation} \textit{#1} \end{quotation} \begin{flushright} - #2, \textit{#3}\end{flushright} }

%%%%%%%%%%%%%%%%%%%%%%%%%%%%%%%%%%%%%%%%%%%%%%%%%%%%%%%
%%%%%% HEADERS AND FOOTERS %%%%%%%%%%%%%%%%%%%%%%%%%%%%
%%%%%%%%%%%%%%%%%%%%%%%%%%%%%%%%%%%%%%%%%%%%%%%%%%%%%%%

% Has been used previously for one-sided layout
% probably doesnt work properly anymore
% \fancypagestyle{onesidedmine}{%
% \fancyhf{} % clear all header and footer fields
% \fancyhead[L]{\textsf{\uppercase{\leftmark}}}%
% \fancyhead[R]{\thepage}%
% \fancyfoot[L]{\scriptsize \invnummer}%
% \fancyfoot[R]{\scriptsize Diplomarbeit \authorname}
% \renewcommand{\headrulewidth}{0.4pt} %
% \renewcommand{\footrulewidth}{0.4pt} %
% }

\fancypagestyle{twosidemine}{
\fancyhf{} % clear all header and footer fields
%\fancyhead[RE]{\textsf{\uppercase{\leftmark}}}%
%\fancyhead[LO]{\textsf{\uppercase{\rightmark}}}%
\fancyhead[RE]{\textsf{\nouppercase{\leftmark}}}%updated to make lower case section headings
\fancyhead[LO]{\textsf{\nouppercase{\rightmark}}}%
\fancyhead[LE,RO]{\thepage}%
\fancyfoot[RE,LO]{\scriptsize Nima Afraz}%
\fancyfoot[LE,RO]{\scriptsize PhD Thesis}
\renewcommand{\headrulewidth}{0.4pt} %
\renewcommand{\footrulewidth}{0.4pt} %
}

\fancypagestyle{bibmine}{\fancyhf{} % clear all header and footer fields
\fancyhead[LO,RE]{\textsf{\uppercase{BIBLIOGRAPHY}}}%
\fancyhead[LE,RO]{\thepage}%
\fancyfoot[RE,LO]{\scriptsize Nima Afraz}%
\fancyfoot[LE,RO]{\scriptsize PhD Thesis}
\renewcommand{\headrulewidth}{0.4pt} %
\renewcommand{\footrulewidth}{0.4pt} %
}

\fancypagestyle{blank}{\fancyhf{}
\renewcommand{\headrulewidth}{0pt} %
\renewcommand{\footrulewidth}{0pt} %
}

% Clear Header Style on the Last Empty Odd pages (for two-sided printing}
\makeatletter
\def\cleardoublepage{\clearpage\if@twoside \ifodd\c@page\else%
   \hbox{}%
   \thispagestyle{blank}%              % Empty header styles
   \newpage%
   \pagestyle{mine}
   \if@twocolumn\hbox{}\newpage\fi\fi\fi}
\makeatother

%\fancypagestyle{mine}{\pagestyle{onesidemine}} % one-sided
\fancypagestyle{mine}{\pagestyle{twosidemine}}  % two-sided


% redefine appendixpage to have a blank page followed by another blank page (called by \begin{appendices})
% (without that there will be a header and footer on an even side... strange)
\renewcommand{\appendixpage}{%
    \cleardoublepage%
    \thispagestyle{blank}%
    \vspace*{0.4\textheight}%
    \centerline{\huge%
       \textbf{\textsf{Appendices}}}%
    \cleardoublepage%
}

%%%%%%%%%%%%%%%%%%%%%%%%%%%%%%%%%%%%%%%%%%%%%%%%%%%%%%%
%%%%%% THEOREMS %%%%%%%%%%%%%%%%%%%%%%%%%%%%%%%%%%%%%%%
%%%%%%%%%%%%%%%%%%%%%%%%%%%%%%%%%%%%%%%%%%%%%%%%%%%%%%%

\theoremstyle{plain}
\newtheorem{thm}{Theorem}[section]    % for theorems (Lehr- od. Grundsatze)
\newtheorem{cor}[thm]{Corollary}      % folgesatz, konsequenz, ...
\newtheorem{lem}[thm]{Lemma}          % hilfssatz, meist im beweis eines wichtigeren
\newtheorem{prop}[thm]{Proposition}   % Satz
\theoremstyle{definition}
\newtheorem{defn}[thm]{Definition}    % Definition
\theoremstyle{remark}
\newtheorem{rem}[thm]{Remark}         % Kommentar

%\newcommand{\Definition}[1]{Definition~\ref{def:#1}}

%%%%%%%%%%%%%%%%%%%%%%%%%%%%%%%%%%%%%%%%%%%%%%%%%%%%%%%
%%%%%% MATH DEFINITIONS %%%%%%%%%%%%%%%%%%%%%%%%%%%%%%%
%%%%%%%%%%%%%%%%%%%%%%%%%%%%%%%%%%%%%%%%%%%%%%%%%%%%%%%

\newcommand{\norm}[1]{\left\lVert#1\right\rVert}   % double vertical barred
\newcommand{\abs}[1]{\left\lvert#1\right\rvert}    % single vertical line
\newcommand{\set}[1]{\left\{#1\right\}}            % in curly braces
\newcommand{\seq}[1]{\left<#1\right>}              % in less-than and greater-than
\newcommand{\field}[1]{\mathbb #1}                 % for number spaces
\newcommand{\Real}{\mathbb R}                      % real numbers
\newcommand{\Complex}{\mathbb C}                   % complex numbers
\newcommand{\Int}{\mathbb Z}                       % integer numbers
\newcommand{\Nat}{\mathbb N}                       % natural numbers
\newcommand{\eps}{\varepsilon}                     % always use \varepsilon
\newcommand{\RE}[1]{\Re\left\{#1\right\}}          % \RE{x} is real part of x
\newcommand{\IM}[1]{\Im\left\{#1\right\}}          % \IM{x}
\newcommand{\To}{\longrightarrow}                  % shortcut for long arrow
\newcommand{\E}[1]{\operatorname{E}\left\{#1\right\}} % \E{x} expected value of x
\newcommand{\e}[1]{\operatorname{e}^{#1}}          % \e{x} is exp(x)
\newcommand{\herm}{^{\mathrm{H}}}                  % hermitian transpose
\newcommand{\tran}{^{\mathrm{T}}}                  % transpose
\renewcommand{\Gamma}{\varGamma}                   % always use italic capital Gamma
\newcommand{\pdv}[1]{\frac{\partial}{\partial #1}} % partial derivative
\newcommand{\dv}{\operatorname{d}\!}               % for derivative notation (dx => \dv x)
\renewcommand{\hat}[1]{\widehat{#1}}               % make \hat wider

%  MACROS FOR EASIER ACCESS (define your own here)
\newcommand{\sigfs}{\stackrel{\mathcal{FS}}{\longrightarrow}}   % Fourier series sign
\newcommand{\sigifs}{\stackrel{\mathcal{FS}}{\longleftarrow}}   % inverse
% vertical Fourier series signs
\newcommand{\sigfsud}{\raisebox{2.5ex}{\rotatebox{270}{$\longrightarrow$}}\,\mathcal{FS}}
\newcommand{\sigifsud}{\raisebox{2.5ex}{\rotatebox{270}{$\longleftarrow$}}\,\mathcal{FS}}
% different notation of FS and DFT
\newcommand{\fs}[1]{\mathcal{FS}\left\{#1\right\}}
\newcommand{\dft}[1]{\mathcal{DFT}\left\{#1\right\}}
% add whatever you need here


%%%%%%%%%%%%%%%%%%%%%%%%%%%%%%%%%%%%%%%%%%%%%%%%%%%%%%%
%%%%%% FLOAT PLACEMENT POLICY %%%%%%%%%%%%%%%%%%%%%%%%%
%%%%%%%%%%%%%%%%%%%%%%%%%%%%%%%%%%%%%%%%%%%%%%%%%%%%%%%

\renewcommand\floatpagefraction{.8}  % fraction of page allowed to be covered by floats
\renewcommand\topfraction{0.8}       % fraction of page allowed to be covered by topfloats
\renewcommand\bottomfraction{0.8}    % fraction of page allowed to be covered by bottomfloats
\renewcommand\textfraction{.2}       % min fraction of text per page
\setcounter{totalnumber}{50}         % max number of floats per page
\setcounter{topnumber}{50}           % max number of topfloats per page
\setcounter{bottomnumber}{50}        % max number of bottomfloats per page

%%%%%%%%%%%%%%%%%%%%%%%%%%%%%%%%%%%%%%%%%%%%%%%%%%%%%%%
%%%%%% MISCELLANEOUS %%%%%%%%%%%%%%%%%%%%%%%%%%%%%%%%%
%%%%%%%%%%%%%%%%%%%%%%%%%%%%%%%%%%%%%%%%%%%%%%%%%%%%%%%

\vfuzz2pt % Don't report over-full v-boxes if over-edge is small (2pt)
\hfuzz2pt % Don't report over-full h-boxes if over-edge is small

\titleformat{\chapter}
{\sffamily\huge\bfseries}{\thechapter}{1em}{}

%% Aesthetic spacing redefines that look nicer to me than the defaults.
\setlength{\cftbeforechapskip}{2ex}
\setlength{\cftbeforesecskip}{0.5ex}

%% Use Helvetica-Narrow Bold for Chapter entries
\renewcommand{\cftchapfont}{\sffamily\bfseries}
\renewcommand{\cftsecfont}{\sffamily}
\renewcommand{\cftsubsecfont}{\sffamily}
\renewcommand{\cftchappagefont}{\sffamily\bfseries}
\renewcommand{\cftsecpagefont}{\sffamily}
\renewcommand{\cftsubsecpagefont}{\sffamily}

%% Because of the font change, the page number becomes too large for the
%% horizontal space LaTeX reserves for it by default. Without the following
%% redefines to fix it, this would cause the Chapter entry page numbers
%% to extend a few points into the right margin. The horror!
\makeatletter
\renewcommand{\@pnumwidth}{1.75em}
\renewcommand{\@tocrmarg}{2.75em}
\makeatother

%% set figures path to figures/ and ../figures (for subdirectories)
\graphicspath{
{./}
{figures/}
{../figures/}
}


\makeindex   % build an index (use \index{index expression} in text)

% to notify stuff to be fixed later (use \todo{fix this} in text)
% \newcommand{\todo}[1]{\textcolor{red}{\textbf{**#1**}}}

\newcommand{\ie}{\emph{i.e.}}
\newcommand{\eg}{\emph{e.g.}}
\newcommand{\etal}{\emph{et al.}}
\newcommand{\corr}[2]{{\color{red}[\MakeTextUppercase{#1}]}{\color{green} #2}}

% create an unnumbered chapter with entry to contents and headers/footers
\newcommand{\chapstar}[1]{%
\cleardoublepage \phantomsection%
\addcontentsline{toc}{chapter}{#1}%
 \chapter*{#1}%
\markboth{\uppercase{#1}}{\uppercase{#1}}%
\acresetall}




% NIMA's PACKAGES

\usepackage{tikz-qtree}
\newtheorem{Definition}{Definition}
\newtheorem{Lemma}{Lemma}
\newtheorem{Theorem}{Theorem}
\newtheorem{Corollary}{Corollary}

% \usepackage{multirow}
\usepackage{ifpdf}
\usepackage{algorithmicx}
\usepackage{algpseudocode}
\usepackage[ruled,vlined,linesnumbered]{algorithm2e}
\usepackage{subcaption}
\captionsetup{compatibility=false}
% \usepackage{amssymb}
% \usepackage{array,booktabs}
\usepackage{bbm}
% \usepackage{amsthm}
\usepackage{cancel}
\usepackage{calrsfs}
\usepackage{tabularx}
\usepackage[normalem]{ulem}
\usepackage{xargs}                      % Use more than one optional parameter in a new commands


\usepackage[autostyle=true]{csquotes} % Required to generate language-dependent quotes in the bibliography
% \usepackage[final]{pdfpages}
% \usepackage[printonlyused]{acronym}

% \usepackage{todonotes}
\usepackage[colorinlistoftodos,prependcaption]{todonotes}
\newcommandx{\towrite}[2][1=]{\todo[inline,linecolor=red,backgroundcolor=red!25,bordercolor=red,#1]{#2}}
\newcommandx{\tochange}[2][1=]{\todo[inline,linecolor=blue,backgroundcolor=blue!25,bordercolor=blue,#1]{#2}}
\newcommandx{\torephrase}[2][1=]{\todo[inline,linecolor=OliveGreen,backgroundcolor=OliveGreen!25,bordercolor=OliveGreen,#1]{#2}}
\newcommandx{\toimprove}[2][1=]{\todo[inline,linecolor=violet,backgroundcolor=violet!25,bordercolor=violet,#1]{#2}}


\usepackage[most]{tcolorbox}
\newtcolorbox{boxbox}[2][]{%
  attach boxed title to top center
               = {yshift=-8pt},
  colback      = blue!5!white,
  colframe     = blue!75!black,
  fonttitle    = \bfseries,
  colbacktitle = blue!85!black,
  title        = #2,#1,
  enhanced,
}